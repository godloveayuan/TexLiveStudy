% 这是注释,不会出现在pdf文档里

% ----------------------------------
% 支持中文,使用 ctexart, 编码需要设置为GBK
% 使用ctexart时,\section默认是居住排版的
%-----------------------------------
\documentclass[]{ctexart}	    

% === 设置标题的格式 ===
\ctexset{
	section = {
		format+= \zihao{-4} \heiti \raggedright,	% 控制左对齐
		name={,、},
		number = \chinese{section},
		beforeskip=1.0ex plus 0.2ex minus .2ex,
		afterskip = 1.0ex plus 0.2ex minus .2ex,
		aftername = \hspace{0pt}
	},
	subsection = {
		format+= \zihao{5} \heiti \raggedright,	% 控制左对齐
		name={,、},
		number = \arabic{section},		% 章节编号,阿拉伯数字
		beforeskip=1.0ex plus 0.2ex minus .2ex,
		afterskip = 1.0ex plus 0.2ex minus .2ex,
		aftername = \hspace{0pt}
	}
}

%\documentclass[
%10pt,		% 文字默认字体大小,一般只有10,11,12 pt
%]{article}			% 使用article时,不支持中文

% 编辑中文内容需要引用的宏包
\usepackage{ctex}
% 使用图片
\usepackage{graphicx}

% 指定存放图片的文件夹路径
\graphicspath{{pictures/},{photoes/}}

% 定义新命令
\newcommand{\mycommend}{\textit{\textbf{\textsf{Fancy Text}}}}

% 文章开始
\begin{document} 
% == 1. 中文,使用 ctex包, xelatex 编译模式
	你好,中国
	
% == 2. 数学公式
 	hello $f(x)=3x^2-x-1$
 	$$ f(x)=x^y-z+x$$
 	$$ f(x)=(x-y)^p/(z-x)^1/p $$
 	
% == 3. 字体族设置 (罗马字体,无衬线字体,打印机字体)
 	% 罗马字体
 	\textrm{这些内容用罗马字体显示 Roman Family}
 	
 	% 无衬线字体
 	\textsf{这些内容用无衬线字体显示 Sans Serif Family}
 	
 	% 打印机字体
 	\texttt{这些内容用打印机字体显示 Typewriter Family}
 	
 	% 声明后续字体为罗马字体
 	\rmfamily 声明后续字体都为罗马字体
 	
 	% 声明后续字体都为无衬线字体
 	\sffamily 声明后续字体都为无衬线字体
 	
 	% 声明后续字体都为打印机字体
 	\ttfamily 声明后续字体都为打印机字体
 	
 	% 使用大括号限定字体作用范围
 	{\sffamily 这些内容限定使用无衬线字体}
 	
 	这是后续内容,当遇到其他字体声明时,会启用新的字体声明,否则延续上文声明的字体,
 	
 	% 字体系列设置(粗细、宽度)
 	\textmd{中等尺寸字体,Medium Series}  \\
	\textbf{加粗字体, Boldface Series} 	\\
	% 字体形状设置(直立、斜体、伪斜体、小型大写)
	\textup{直立字体,Upright Shape}		\\
	\textit{斜体,Italic Shape}		\\
	\textsl{伪斜体,Slanted Shape}	\\
	\textsc{小型大写,Small Caps Shape}	\\
	% 设置影响后续内容
	{\upshape 影响后续内容为直立字体,大括号限制作用域 Upright Shape} \\
	{\itshape 影响后续内容为斜体字体,大括号限制作用域 Italic Shape} \\
	{\slshape 影响后续内容为伪斜体字体,大括号限制作用域 Slanted Shape} \\
	{\scshape 影响后续内容为小型大写字体,大括号限制作用域 Small Caps Shape} \\
	% 中文字体
	{\songti 宋体} \quad 
	{\heiti 黑体} \quad
	{\fangsong 仿宋} \quad
	{\kaishu 楷书} \quad 	
	\quad 是空格 \\
	{\textbf 中文粗体使用黑体表示}\\
	{\textit 中文斜体使用楷书表示} \\
	
	% 字体大小,相对于 \documentclass 声明中的 normal 字体大小
	{\tiny 			hello}\\
	{\scriptsize 	hello}\\
	{\footnotesize	hello}\\
	{\small			hello}\\
	{\normalsize	hello}\\
	{\large			hello}\\
	{\Large			hello}\\
	{\LARGE			hello}\\
	{\huge			hello}\\
	{\Huge			hello}\\
	
	% 中文设置字号
	\zihao{-0} 初号字 \\
	\zihao{5} 5号字   \\
	
	% 使用自定义命令
	\mycommend
	
% == 4. 章节结果
	% 产生目录
	\tableofcontents

	% section 构建章节, 加*号去掉自动编号
	\section*{引言}
	\subsection*{小节}
	这是引言的内容\\ 这是换行,首行并不缩进,换行符合不能出行在一行的开头。
	
	使用空行也能换行,一个空行或多个空行效果是一样的,这种换行会首行缩进。
	
	\paragraph{段落标题} 产生新的段落
	
	\section{第一章}
	% subsection 构建小节
	\subsection{第一小节}
	\subsubsection{三级小节}
	
	\section{第二章}
	\subsection{第一小节}
	
	\section{第三章}
	\subsection{第一小节}
	
	% \documentclass 是 texbook 时可以使用 \chapter{} 命令
 	

% == 5. 特殊字符
	\section{空白符}
	% 空行分段:多个空行等同与一个
	% 自动缩进:绝对不能使用空格代替
	% 空格:英文中多个处理为1个,中文中空格被忽略
	% 汉字与其他字符的间距会自动有 XeLaTex 处理
	% 禁止使用中文全角空格
	% 使用空白符时,如下:
	a\quad b	% 1em
	a\qquad b	% 2em
	a\thinspace b 或 a\,b	% 约位 1/6个em
	a\enspace b	% 0.5em
	a~b			% 硬空格
	% 1pc=12pt=4.218em
	a\kern 1pc b
	a\kern -1em b
	a\hskip 1emb
	a\hspace{35pt}b
	
	a\hphantom{xyz}b	%占位宽度
	a\hfill b			% 弹性长度
	
	\section{控制符}
	% & #、$、%、{、}、~、等符号都有特殊意义,要输入原符号时需要加反斜线转义
	\# \$ \{ \} \_{} \~{} \^{} 
	这是一个换行符号:\textbackslash 
	
	\&

	\section{排版特殊符号}
	\S \P \dag \ddag \copyright \pounds
	
	\section{\TeX 标志符号}
	\TeX{}
	\LaTeX{}
	\LaTeXe{}
	
% == 6. 插入图片
	\section{插入图片}
	% 使用 \usepackage{graphicx} 宏包
	% 导言区: \usepackage{graphicx}
	% 语  法: \includegraphics[<选项>]{<文件名>}
	% 文件格式:\ EPS, PDF,PNG, JPEG, BMP
	\includegraphics{imgtest}	% 可以省略文件后缀名
	% 通过可选参数指定缩放、高度
	\includegraphics[
	%width=13cm,		% 图片宽度
	%height=8cm,		% 图片高度
	%width=0.1\textwidth,	% 相对宽度
	%height=0.1\textheight,	% 相对高度
	scale=0.5,			% 图片缩放
	]{imgtest.jpg}
	
% == 7. 表格
	\section{插入表格}
	% 使用 tabular生成表格
	% 生成5列的表格 l:左对齐,c:居中对齐, r:右对齐 , | 产生表格竖线, || 产生双竖线
	% p 指定列宽度
	\begin{tabular}{l || c  c | p{1.5cm} | r}
		\hline   % 产生表格横线	
		姓名 & 语文 & 数学 & 外语 & 备注 \\ % 用 & 作为列分隔符 ,\\ 产生新行	
		\hline \hline  % 产生双横线
		张三 & 100 & 100 & 100 & 优秀 \\	
		\hline   % 产生表格横线
	\end{tabular}

	% 使用 booktab 宏包实现三线表
	% 使用 longtab 宏包实现跨页长表格
	
	
% == 8. 浮动环境
	% 使用浮动体可以:实现灵活分页
	% 给图片添加标题
	% 引用图表编号
	% 排版参数:
	% h : 此处(here) 代码所在的上下文位置
	% t : 页顶(top), 代码所在页或之后页的顶部
	% b : 页底(bottom), 代码所在页或之后页的底部
	% p : 独立一页(page)

	% 图片浮动
	\section{图片浮动}
	想看美女吗,见图\ref{mm}
	\begin{figure}[htbp]
		\centering			% 居中
		
		\includegraphics[
		%width=13cm,		% 图片宽度
		%height=8cm,		% 图片高度
		%width=0.1\textwidth,	% 相对宽度
		%height=0.1\textheight,	% 相对高度
		scale=0.5,			% 图片缩放
		]{imgtest.jpg}
		\caption{美女}		% 设置图片标题,命令在图片下则标题在图片下,自动编号
		\label{mm}			% 生成标签
	\end{figure}
	
	% 表格浮动
	\section{表格浮动}
	想看成绩吗 ,见表 \ref{firsttable}
	\begin{table}[htbp]		% [位置参数]
		\centering			% 居中
		\caption{成绩表}		% 设置表格标题,自动编号
		\begin{tabular}{l || c  c | p{1.5cm} | r}
			\hline   % 产生表格横线	
			姓名 & 语文 & 数学 & 外语 & 备注 \\ % 用 & 作为列分隔符 ,\\ 产生新行	
			\hline \hline  % 产生双横线
			张三 & 100 & 100 & 100 & 浮动表格 \\	
			\hline   % 产生表格横线
		\end{tabular}
		\label{firsttable}
	\end{table}
 	
 % 文章结束
 \end{document}